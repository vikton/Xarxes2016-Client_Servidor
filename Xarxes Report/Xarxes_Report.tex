\documentclass[11pt]{article}
\usepackage{ragged2e}
\usepackage[utf8]{inputenc}
\usepackage[catalan]{babel}
\usepackage{hyperref}
\usepackage{graphicx}
\graphicspath{ {Images/} }

\begin{document}
\begin{titlepage}

\newcommand{\HRule}{\rule{\linewidth}{0.5mm}} % Defines a new command for the horizontal lines, change thickness here

\center % Center everything on the page
 
%----------------------------------------------------------------------------------------
%	HEADING SECTIONS
%----------------------------------------------------------------------------------------

\textsc{\LARGE Universitat de Lleida}\\[1.5cm] % Name of your university/college
\includegraphics{Images/logoUDL.jpg}\\[1cm] % Include a department/university logo - this will require the graphicx package
\textsc{\Large Enginyeria Informàtica}\\[0.5cm] % Major heading such as course name
\textsc{\large Xarxes}\\[0.5cm] % Minor heading such as course title

%----------------------------------------------------------------------------------------
%	TITLE SECTION
%----------------------------------------------------------------------------------------

\HRule \\[0.4cm]
{ \huge \bfseries Pràctica 1, Programació d’aplicacions de xarxa}\\[0.4cm] % Title of your document
\HRule \\[1.5cm]
 
%----------------------------------------------------------------------------------------
%	AUTHOR SECTION
%----------------------------------------------------------------------------------------

\begin{minipage}{0.4\textwidth}
\begin{flushleft} \large
\emph{Author:}\\
Jordi Ricard Onrubia Palacios % Your name
\end{flushleft}
\end{minipage}
~
\begin{minipage}{0.4\textwidth}
\begin{flushright} \large
\emph{Professor:} \\
Enric Guitart % Supervisor's Name
\end{flushright}
\end{minipage}\\[4cm]

% If you don't want a supervisor, uncomment the two lines below and remove the section above
%\Large \emph{Author:}\\
%John \textsc{Smith}\\[3cm] % Your name

%----------------------------------------------------------------------------------------
%	DATE SECTION
%----------------------------------------------------------------------------------------

{\large \today}\\[3cm] % Date, change the \today to a set date if you want to be precise


\vfill % Fill the rest of the page with whitespace

\end{titlepage}
\newpage
\section*{Resum}
\justify
\newpage
\tableofcontents
\newpage
\section{Client}
\justify
Programat en Python versió 2.7.10.
\\\\
El client és un programa què es comunicarà amb el servidor, inicialment, aquest demana permís per a connectar-se, un cop connectat mantindrà una comunicació temporitzada amb el servidor per tal d'informar de la seva presencia i evitar la desconnexió amb el servidor.
%Diagrama estats
\\\\
A més a més, aquest podrà rebre una sèrie de comandes per tal d'enviar o rebre el seu arxiu de configuració.
\\\\
A continuació s'expliquen les funcions que ha de realitzar i la seva resolució.

	\subsection{Registrar-se al servidor}
Per a la realització del registre caldra omplir la PDU (Protocol Data Unit) amb els camps que s'indiquen a continuació:
%IMAGEN PDU




	\subsection{Manteniment de communicació}
	\subsection{Esperar comandes per l'enviament/recepció del seu arxiu de configuració}
	\subsection{Funcions adicionals}
	
\section{Servidor}
	\subsection{Enregistrament de clinets}
	\subsection{Manteniment de comunicació amb equips enregistrats }
	\subsection{Enviament/recepció de arxius de configuració}
	\subsection{Espera de comandes per consola}
\section{Anexe}
	\subsection{Problemes i solucions trobats en el desenvolupament}
		\subsubsection*{Client}
		\subsection*{Servidor}
\begin{enumerate}
\item Si els dos primers ''ALIVES'' la temporització no arriba a agafar el tercer ''ALIVE'' en cas de que fos enviat ja que aquest arriba just en el moment en que el servidor a desconectat al client. La solució ha estat aplicar un temporitzador per tal de que el servidor només comprovi els estats dels clients cada segon en lloc de continuament, a més a més, es permet que el temps en el que tindria que arribar ''l'ALIVE'' sigui més gran que l'especificat, exactament 1 segon més.
\end{enumerate}
\newpage
\section{Referències}
Importar constants d'un altre fitxer en Python 2.\\
\url{http://zetcode.com/lang/python/packages/}\\
Structs en Python 2.\\
\url{https://docs.python.org/2/library/struct.html}\\
UDP sockets en Python 2.\\
\url{https://wiki.python.org/moin/UdpCommunication}\\
Funcions de la llibreria Time de Python 2.\\	
\url{https://docs.python.org/2/library/time.html}\\
TCP sockets en Python 2.\\
\url{https://wiki.python.org/moin/TcpCommunication}\\
Signals en Python 2.\\
\url{https://docs.python.org/2/library/signal.html}\\
UDP sockets en C.\\
\url{https://www.cs.rutgers.edu/~pxk/417/notes/sockets/udp.html}\\
Funcions de la llibreria Time de C.\\
\url{http://www.cplusplus.com/reference/ctime/}
\end{document}