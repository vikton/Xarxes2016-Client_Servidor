\documentclass[11pt]{article}
\usepackage{ragged2e}
\usepackage[utf8]{inputenc}
\usepackage[catalan]{babel}
\usepackage{hyperref}
\usepackage{graphicx}
\usepackage{array}
\usepackage{afterpage}
\graphicspath{ {Images/} }

\begin{document}
\begin{titlepage}

\newcommand{\HRule}{\rule{\linewidth}{0.5mm}} % Defines a new command for the horizontal lines, change thickness here

\center % Center everything on the page
 
%----------------------------------------------------------------------------------------
%	HEADING SECTIONS
%----------------------------------------------------------------------------------------

\textsc{\LARGE Universitat de Lleida}\\[1.5cm] % Name of your university/college
\includegraphics{Images/logoUDL.jpg}\\[1cm] % Include a department/university logo - this will require the graphicx package
\textsc{\Large Grau en Enginyeria Informàtica}\\[0.5cm] % Major heading such as course name
\textsc{\large Xarxes}\\[0.5cm] % Minor heading such as course title

%----------------------------------------------------------------------------------------
%	TITLE SECTION
%----------------------------------------------------------------------------------------

\HRule \\[0.4cm]
{ \huge \bfseries Pràctica 1, Programació d’aplicacions de xarxa}\\[0.4cm] % Title of your document
\HRule \\[1.5cm]
 
%----------------------------------------------------------------------------------------
%	AUTHOR SECTION
%----------------------------------------------------------------------------------------

\begin{minipage}{0.4\textwidth}
\begin{flushleft} \large
\emph{Autor:}\\
Jordi Ricard Onrubia Palacios % Your name
\end{flushleft}
\end{minipage}
~
\begin{minipage}{0.4\textwidth}
\begin{flushright} \large
\emph{Professor:} \\
Enric Guitart Baraut % Supervisor's Name
\end{flushright}
\end{minipage}\\[4cm]

%----------------------------------------------------------------------------------------
%	DATE SECTION
%----------------------------------------------------------------------------------------
{\large \today}\\[3cm] % Date, change the \today to a set date if you want to be precise
\vfill % Fill the rest of the page with whitespace
\end{titlepage}

%%%%%%%%%%%%%%%%%%%%%%%%%%%%%%%%%
%%%%%%%%%%%%ABSTRACT%%%%%%%%%%%%%
%%%%%%%%%%%%%%%%%%%%%%%%%%%%%%%%%
\newpage
\section*{Resum}
En el següent document es troba la informació necessària per a comprendre el funcionament del client i el servidor proposats per la pràctica. Per a la seva explicació es fan ús de diagrames d'estats on es mostra el funcionament dels protocols emprats. Per a mostrar la implantació del client i el servidor, s'explique l'estructuració juntament amb diagrames d'execució, on es mostren les funcions utilitzades i com s'executen entre elles.
\justify
\thispagestyle{empty}
%%%%%%%%%%%%%%%%%%%%%%%%%%%%%%%%%
%%%%%%%%%%%%INDEX%%%%%%%%%%%%%%%%
%%%%%%%%%%%%%%%%%%%%%%%%%%%%%%%%%
\newpage
\thispagestyle{empty}
\tableofcontents
%%%%%%%%%%%%%%%%%%%%%%%%%%%%%%%%%
%%%%%%%%%%%%COS%%%%%%%%%%%%%%%%%%
%%%%%%%%%%%%%%%%%%%%%%%%%%%%%%%%%
\newpage
\clearpage
\pagenumbering{arabic}
\section{Client}
\subsection{Protocol UDP}	
El protocol UDP(\textit{User Datagram Protocol}) en la fase del client és utilitzat per a establir i mantenir la comunicació entre ell i el servidor, així doncs els estats del client depenen del missatges transmessos per aquest.\\\\
En la següent figura es mostra un diagrama amb el seu funcionament principal.
\\\\
\begin{figure}[h]
    \centering
    \includegraphics[width=0.8\textwidth]{UDP_diagram.png}
    \caption{Protocol UDP}
    \label{fig:ProtcolUDP}
\end{figure}
\newpage
\subsection{Protocol TCP}
El protocol TCP(\textit{Transmission Control Protocol}) es utilitzat per a realitzar el procés de enviament i petició del arxiu de configuració del client cap al servidor.
\\\\
En la següent figura es mostra un diagrama amb el seu funcionament principal.
\\\\
\begin{figure}[h]
    \centering
    \includegraphics[width=0.8\textwidth]{clientTCP.png}
    \caption{Client: Protocol TCP}
    \label{fig:PrtocolTCP}
\end{figure}
\newpage
\justify
\subsection{Estructura}
Programat en: Python versió 2.7.10.\\\\
El client està dividit en 2 fases diferents:
\subsection*{Primera fase: Registre}
Per a la primera fase el client al iniciar-se fa un intent de registre al servidor mitjançant el protocol UDP, un cop finalitzant l'intent de registre passem a la següent fase.
\\\\
Com es pot observar en la figura \ref{fig:Client}\footnote{En aquest esquema no es mostren les operacions opcionals com el mode debug, el canvi de l'arxiu d'arrancada o el de configuració encara que hi estan presents, ja que només mostra les funcions principals i imprescindibles.} la comunicació entre les funcions que intervenen en la primera fase són aquelles numerades de l'1 al 18.
\subsection*{Segona fase: Manteniment de comunicació i espera de comandes per teclat}
En la segona fase intervenen processos el primer encarregat de mantenir la comunicació i el segon encarregat de rebre comandes per teclat, per tal que aquests processos facin les seves funcions previament caldrà comprovar que el registre ha estat realitzat correctament.\\\\
La comunicació entre les funcions que intervenen en la comprovació correcta del registre són aquelles numerades del 19 al 24.\\\\
La comunicació entre les funcions que intervenen en l'espera de comandes per teclat és aquella numerada amb el 25 on es crea la bifurcació entre els processos i aquelles numerades del 26 al 40.\\\\
La comunicació entre les funcions que intervenen en el manteniment de la comunicació és aquella numerada amb el 25 on es crea la bifurcació entre els processos i aquelles numerades del 41 al 45.
\afterpage{\clearpage}
\begin{figure}[h]
    \includegraphics[width=0.8\textwidth]{Client.png}
    \caption{Client}
    \label{fig:Client}

\end{figure}
\newpage
\section{Servidor}
Compilat en: Linux versió 4.2.0 gcc versió 5.2.1.\\
Comanda utilitzada: gcc -ansi -pedantic -Wall.\\
El servidor no utilitza Protocol UDP més que per a controlar els estats del client per tant el protocol és el mateix que l'indicat en la figura \ref{fig:ProtcolUDP}
\subsection{Protocol TCP}
Com en al client, el protocol TCP també hi és present al servidor, el procés és el mateix amb la diferencia que el servidor ha de escoltar i acceptar o denegar les conexions que arribin per part dels clients.
\begin{figure}[h]
    \centering
    \includegraphics[width=0.8\textwidth]{ServidorTCP.png}
    \caption{Servidor: Protocol TCP}
    \label{fig:PrtocolTCPserv}
\end{figure}
\newpage

\subsection{Estructura}
El servidor al contrari del client no te diferentes fases ja que ha de realitzar les accions de rebre registres, contestar als missatges de manteniment de comunicació, rebre o enviar els fitxers de configuració en cas de que un client faci una petició i atendre a les comandes rebudes per consola totes a la vegada, per tant el dividim en processos.\\\\
Processos:

\begin{enumerate}
\item
Prepara el servidor per a iniciar-se i es queda a la espera de rebre comandes, com es pot observar a la figura \ref{fig:server} la comunicació la comporten les funcions numerades del 1 al 7.
\item
Creará el socket UDP i creará dos fills mès, la comunicació la comporten les funcions numerades del 8 i 9.
\item
Encarregat de rebre peticions i crear nous processos per a realitzar-les, aquests processos creats moriràn un cop finalitzada la petició, la comunicació la comporten les funcions numerades del 14 al 28.
\item
Comprova cada segón la temporització dels 'ALIVES' dels equips registrats.
\end{enumerate}
\newpage
\begin{figure}[h]
    \centering
    \includegraphics[width=0.8\textwidth]{server.png}
    \caption{Servidor}
    \label{fig:server}
\end{figure}
%%%%%%%%%%%%%%%%%%%%%%%%%%%%%%%%%
%%%%%%%%%%%%ANEXE%%%%%%%%%%%%%%%%
%%%%%%%%%%%%%%%%%%%%%%%%%%%%%%%%%
\newpage
\section{Anexe}
	\subsection{Execució}
		\subsubsection*{Client:}
Execució:
\begin{itemize}
\item Executar el fitxer client.py de la següent manera: python client.py
\item És important que el fitxer cons.py estigui en el mateix arxiu que client.py, ja que aquest conté totes les constants que han de ser utilitzades en l'execució del client.
\end{itemize}
\subsubsection*{Servidor:}
Compilació:
\begin{itemize}
\item Executar el fitxer Makefile inclòs de la següent manera: ./Makefile.
\item Si el fitxer Makefile no pot ser executat per falta de permisos executar la comanda: chmod +x Makefile.
\item És important que el fitxer server.c estigui en el mateix arxiu que el Makefile.
\end{itemize}
Execució:
\begin{itemize}
\item Executar el fitxer server generat per Makefile després de la seva execució de la següent manera: ./server
\item Si el fitxer server no pot ser executat per falta de permisos executar la comanda: chmod +x server.
\end{itemize}
\newpage
\subsection{Problemes i solucions trobats en el desenvolupament}
\subsubsection*{Client}
\justify
\begin{enumerate}
\item Bloqueig del programa per part del recvfrom, recvfrom és una funció què s'encarrega de rebre el missatge rebut pel socket, la solució aplicada va ser posar el timeout a 0 fent-la així no bloquejant.
\item Senyal SIGTERM utilitzada per finalitzar el client a l'hora de rebre 'quit' ignorada, la solució aplicada va ser fer enviaments fins que un d'aquests no sigui ignorat i finalitzi el client.
\item Llançament de excepcions per part de Python a l'hora de detectar per teclat un ctrl-c, la solució aplicada va ser tractar la lectura del teclat amb un try except ignorant les interrupcions de teclat.
\end{enumerate}
\subsection*{Servidor}
\begin{enumerate}
\item Problema d'accedir i actualitzar les dades dels clients des d'altres processos. La solució ha estat fer un mmap utilitzant la llibreria sys/man.h.
\item Si els dos primers ''ALIVES'' la temporització no arriba a agafar el tercer ''ALIVE'' en cas que fos enviat, ja que aquest arriba just en el moment en què el servidor ha desconnectat al client. La solució ha estat aplicar un temporitzador per tal de què el servidor només comprovi els estats dels clients cada segon en lloc de contínuament, a més a més, es permet que el temps en què haguera d'arribar ''l'ALIVE'' sigui més gran que l'especificat, exactament 1 segon més.
\end{enumerate}
	\subsection{Aspectes a tenir en compte}
	\subsubsection*{Client}
\begin{itemize}
\item Qualsevol error en l'apartat de registre el fará desconnectar completament, no iniciarà un altre procés de registre.
\item A l'hora de cridar el progrma amb '-c' està permés introduir noms de fitxer erronis, però, a l'hora de fer la lectura donarà missatge de error i es tancarà el servidor, si es crida el programa amb '-f' i s'introdueix un nom de fitxer erroni no es permetrá realitzar l'intercanvi del fitxer de configuració amb el servidor.
\end{itemize} 
	 \subsubsection*{Servidor}
\begin{itemize}
\item A l'hora de cridar el servidor amb '-u' o '-c' està permés introduir noms de fitxer erronis, però, a l'hora de fer la lectura donarà missatge de error i es tancarà el servidor.
\item Un equip que s'ha desconectat per qualsevol motiu no pot tornar a registrar-se instantaneament, ha de passar un segón.
\item Per poder realitzar 'get-conf' o 'send-conf' és necessari que el client hagi realitzat satisfactoriament el primer 'ALIVE'.
\item Si es rep una petició de enviament de fitxer de configuració i aquest no existeix no s'enviará al client.
\end{itemize}	
	
	\subsection{Llistat de figures}
\listoffigures{}
\newpage
\section{Bibliografia}
Importar constants d'un altre fitxer en Python 2.\\
\url{http://zetcode.com/lang/python/packages/}\\
Structs en Python 2.\\
\url{https://docs.python.org/2/library/struct.html}\\
UDP sockets en Python 2.\\
\url{https://wiki.python.org/moin/UdpCommunication}\\
Funcions de la llibreria Time de Python 2.\\	
\url{https://docs.python.org/2/library/time.html}\\
TCP sockets en Python 2.\\
\url{https://wiki.python.org/moin/TcpCommunication}\\
Signals en Python 2.\\	
\url{https://docs.python.org/2/library/signal.html}\\
UDP sockets en C.\\
\url{https://www.cs.rutgers.edu/~pxk/417/notes/sockets/udp.html}\\
TCP sockets en C\\
\url{http://www.linuxhowtos.org/C_C++/socket.htm}	\\
Compartir memória en C.\\
\url{http://man7.org/linux/man-pages/man2/mmap.2.html}\\
Funcions de la llibreria Time de C.\\
\url{http://www.cplusplus.com/reference/ctime/}\\
\end{document}