\documentclass[11pt]{article}
\usepackage{ragged2e}
\usepackage[utf8]{inputenc}
\usepackage[catalan]{babel}
\usepackage{hyperref}
\usepackage{graphicx}
\graphicspath{ {Images/} }

\begin{document}
\title{Xarxes: Pràctica 1, Programaci\'{o} d\textsc{\char13} aplicacions de xarxa}
\author{Jordi Ricard Onrubia Palacios}
\date{\today}
\maketitle
\newpage
\section*{Resum}
\justify
\newpage
\tableofcontents
\newpage
\section{Client}
\justify
Programat en Python versió 2.7.10.
\\\\
El client és un programa què es comunicarà amb el servidor, inicialment, aquest demana permís per a connectar-se, un cop connectat mantindrà una comunicació temporitzada amb el servidor per tal d'informar de la seva presencia i evitar la desconnexió amb el servidor.
\\\\
A més a més, aquest podrà rebre una sèrie de comandes per tal d'enviar o rebre arxius de configuració.
\\\\
A continuació s'expliquen les funcions que ha de realitzar i la seva resolució.

	\subsection{Registrar-se al servidor}
Per a la realització del registre caldra omplir la PDU (Protocol Data Unit) amb els camps que s'indiquen a continuació.



	\subsection{Manteniment de communicació}
	\subsection{Esperar comandes per l'enviament/recepció del seu arxiu de configuració}
	\subsection{Funcions adicionals}
	
\section{Servidor}
	\subsection{Enregistrament de clinets}
	\subsection{Manteniment de comunicació amb equips enregistrats }
	\subsection{Enviament/recepció de arxius de configuració}
	\subsection{Espera de comandes per consola}
\section{Apèndix}
\section{Referències}
\end{document}